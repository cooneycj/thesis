\documentclass[12pt]{article}
%\usepackage{fullpage}
\linespread{1.5}
\usepackage{helvet}
\fontfamily{phv}\selectfont
\pagestyle{empty}
\pagenumbering{gobble}
\begin{document}

\title{{\bf Abstract} \\ Spike Train Analysis}
\author{Cathal Cooney}
\date{29th May 2015}
\maketitle

This thesis provides several new approaches to the study of spike trains - the electrical signals which neurons produce to communicate with each other.

Spike train metrics are central to the study of spike train analysis, and several extensions are proposed for these metrics for use in multi-unit recordings.  A parameter-free multi-unit metric is introduced, which could be used on all neural recordings without the need to tune parameters.

Clustering methods from complex network theory were also used, along with information theory measures, were employed to map cooperative groups of neurons. 

Finally, a simple neuron model was introduced.  A bimodal Poisson rate process is suggested motivated by feature-specific neurons.  This model leads to a calculation of the inter-spike interval (ISI) distribution.  Tested on data, there appears to be evidence of such bimodality.

\end{document}
