The SPIKE distance proposed by Kreuz et al in \cite{Kreuzetal2010} is an instantaneous parameter-free distance measure between spike-trains.  By instantaneous measure, it is meant that for each time $t$ there is a time-local distance between two spike-trains.  This measure can the be integrated over the length of the spike-trains to give a parameter-free distance measure between spike trains. In the study of spike-train metrics, often a multi-unit measure is desired (to minimise compounding any errors that may result from inefficient spike-sorting?). A multi-unit measure is a distance between collections of labelled spike-trains.  The SPIKE distance in \cite{Kreuzetal2010} does not lend itself to a simple multi-unit extension, so instead a simpler version of the SPIKE distance is extended.

\section{Single-unit recordings}
With a view to extending the SPIKE distance to a multi-unit distance measure, it is useful to review the definition of the SPIKE measure provided in \cite{Kreuzetal2010}:  Given two spike-trains $x$ and $y$, where $x = \{ t_1^x, \ldots, t_n^x \}$ and $y = \{ t_1^y, \ldots , t_m^y\}$, where $t_1^x,\ldots,t_n^x,t_1^y,\ldots,t_m^y$ are the spike times.  The SPIKE distance in \cite{Kreuzetal2010} has the nice property that the distance is bounded such that it is always between zero and one.  Unfortunately, to achieve a "natural" extension to a multi-unit measure, this property is sacrificed.  A simpler version of the distance, without the strict upper bound of one, is used.

The simpler distance is very quick and easy to calculate.  A `gap' is associated with each spike; this is the distance to the nearest spike in the other spike-train.  The total distance between two spike trains is then the sum of these gaps.  The original SPIKE distance included additional normalisation factors that have been omitted in simplifying the measure.  To form the time-local distance, a weighted sum of the gaps for the corner spikes is made, that is, the spikes preceding and following the time of interest in each of the two spike trains.  The weighting is chosen so that the integral of the time-local function is just the sum of the gaps.

First, the gaps are calculated for each spike in the two spike-trains.  This is simply the nearest spike in the other spike train:
\begin{equation}
\Delta t_i^x = min_i ( | t_i^x - t_i^y |)
\end{equation}
At each time instant, these is a unique set of four corner spikes: the preceding and following spikes from each spike train, which are labelled $t_P^x(t), t_F^x(t), t_P^y(t)$ and  $t_F^y(t)$; these are, respectively, the preceding and following spikes in spike-train $x$ and the preceding and following spikes in spike-train $y$.

For each spike-train, w, a time-local distance is then calculated using the associated gap of the four corner spikes for each spike-train, $w=x,y$:
\begin{equation}
s_w(t) = \frac{\lambda_P(t) \Delta t_P^w(t) + \lambda_F(t)\Delta t_F^w(t)}{I^w(t)}
\end{equation}
where
\begin{equation}
\lambda_F^w(t) = \frac{t- t_P^w(t)}{I^w(t)}, \, \lambda_P^w(t) = \frac{ t_F^w(t) - t}{I^w(t)}
\end{equation}
and $I^w(t)$ is the size of the interval in which $t$ is contained:
\begin{equation}
I^w(t) = t_F^w(t) - t_P^w(t).
\end{equation}
Now, the time-local distance for each neuron is added to give the overall time-local distance:
\begin{equation}
s(t) = s_x(t) + s_y(t).
\end{equation}

This simplified time-local SPIKE distance has the advantage that its integral is simply the sum of the gaps of each spike; that is:
\begin{equation}
\int_0^T s(t)\, dt = \sum_w \sum_i \Delta t_i^w.
\end{equation}

In practice this simplified version of SPIKE produces similar time profiles to the version described in \cite{Kreuzetal2010}. The time-local distance profile for two similar spike-trains is given in Figure [NEED TO ADD PICTURE].

\section{Extension to multi-unit recordings}